% Options for packages loaded elsewhere
\PassOptionsToPackage{unicode}{hyperref}
\PassOptionsToPackage{hyphens}{url}
%
\documentclass[
]{article}
\usepackage{amsmath,amssymb}
\usepackage{iftex}
\ifPDFTeX
  \usepackage[T1]{fontenc}
  \usepackage[utf8]{inputenc}
  \usepackage{textcomp} % provide euro and other symbols
\else % if luatex or xetex
  \usepackage{unicode-math} % this also loads fontspec
  \defaultfontfeatures{Scale=MatchLowercase}
  \defaultfontfeatures[\rmfamily]{Ligatures=TeX,Scale=1}
\fi
\usepackage{lmodern}
\ifPDFTeX\else
  % xetex/luatex font selection
\fi
% Use upquote if available, for straight quotes in verbatim environments
\IfFileExists{upquote.sty}{\usepackage{upquote}}{}
\IfFileExists{microtype.sty}{% use microtype if available
  \usepackage[]{microtype}
  \UseMicrotypeSet[protrusion]{basicmath} % disable protrusion for tt fonts
}{}
\makeatletter
\@ifundefined{KOMAClassName}{% if non-KOMA class
  \IfFileExists{parskip.sty}{%
    \usepackage{parskip}
  }{% else
    \setlength{\parindent}{0pt}
    \setlength{\parskip}{6pt plus 2pt minus 1pt}}
}{% if KOMA class
  \KOMAoptions{parskip=half}}
\makeatother
\usepackage{xcolor}
\usepackage[margin=1in]{geometry}
\usepackage{graphicx}
\makeatletter
\def\maxwidth{\ifdim\Gin@nat@width>\linewidth\linewidth\else\Gin@nat@width\fi}
\def\maxheight{\ifdim\Gin@nat@height>\textheight\textheight\else\Gin@nat@height\fi}
\makeatother
% Scale images if necessary, so that they will not overflow the page
% margins by default, and it is still possible to overwrite the defaults
% using explicit options in \includegraphics[width, height, ...]{}
\setkeys{Gin}{width=\maxwidth,height=\maxheight,keepaspectratio}
% Set default figure placement to htbp
\makeatletter
\def\fps@figure{htbp}
\makeatother
\setlength{\emergencystretch}{3em} % prevent overfull lines
\providecommand{\tightlist}{%
  \setlength{\itemsep}{0pt}\setlength{\parskip}{0pt}}
\setcounter{secnumdepth}{-\maxdimen} % remove section numbering
\ifLuaTeX
  \usepackage{selnolig}  % disable illegal ligatures
\fi
\usepackage{bookmark}
\IfFileExists{xurl.sty}{\usepackage{xurl}}{} % add URL line breaks if available
\urlstyle{same}
\hypersetup{
  pdftitle={Analise INSE 2021},
  pdfauthor={Vinicius Ferreira Git-VFS10 Insta-Vinifersan89},
  hidelinks,
  pdfcreator={LaTeX via pandoc}}

\title{Analise INSE 2021}
\author{Vinicius Ferreira Git-VFS10 Insta-Vinifersan89}
\date{2024-05-23}

\begin{document}
\maketitle

\subsubsection{Objetivo !}\label{objetivo}

Neste relatório iremos Analise o Nível SocioEconômico Brasileiro de 2021
(INSE)

\subsubsection{Oque é o INSE ?}\label{oque-uxe9-o-inse}

\textbf{O Indicador de Nível Socioeconômico (Inse) é uma medida
utilizada para classificar as escolas públicas do Brasil.} O Inse é
calculado com base em dados do Censo Escolar, como a escolaridade dos
pais, a posse de bens e serviços da família e o acesso à internet.

\textbf{O Inse é usado para contextualizar o desempenho das escolas nas
avaliações e exames realizados pelo Instituto Nacional de Estudos e
Pesquisas Educacionais Anísio Teixeira (Inep).} Isso permite que o Inep
compare o desempenho das escolas de diferentes níveis socioeconômicos e
identifique as escolas que precisam de mais apoio.

\emph{O Inse é uma ferramenta importante para a melhoria da educação no
Brasil.} Ele ajuda a identificar as escolas que precisam de mais apoio e
permite que o INEP desenvolva políticas públicas para melhorar a
educação de todos os alunos, independentemente de sua origem social.

\textbf{\emph{OS dados de INSE são providenciados pelo Sistema Nacional
de Avaliação da Educação Básica (Saeb).}}

\textbf{O Sistema Nacional de Avaliação da Educação Básica (SAEB) é um
sistema de avaliação externa em larga escala, composto por um conjunto
de instrumentos, realizado periodicamente pelo Instituto Nacional de
Estudos e Pesquisas Educacionais Anísio Teixeira (INEP),} desde os anos
1990, e tem por objetivos, no âmbito da educação básica: (I) produzir
indicadores educacionais para o Brasil, suas regiões e unidades da
Federação e, quando possível, para os municípios e as instituições
escolares, tendo em vista a manutenção da comparabilidade dos dados,
permitindo, assim, o incremento das séries históricas; (II) avaliar a
qualidade, a equidade e a eficiência da educação praticada no País em
seus diversos níveis governamentais; (III) subsidiar a elaboração, o
monitoramento e o aprimoramento de políticas públicas em educação
baseadas em evidências, com vistas ao desenvolvimento social e econômico
do Brasil; (IV) desenvolver competência técnica e científica na área de
avaliação educacional, ativando o intercâmbio entre instituições de
ensino e pesquisa (Brasil. Inep, 2021b).

\begin{verbatim}
## Carregando pacotes exigidos: gsubfn
\end{verbatim}

\begin{verbatim}
## Carregando pacotes exigidos: proto
\end{verbatim}

\begin{verbatim}
## Carregando pacotes exigidos: RSQLite
\end{verbatim}

\textbf{Média de INSE das familias dos estados}

Ao realizarmos a plotagem da média INSE das familias dos estados do
Brasil em um gráfico de barras, identificamos que o estado com maior
indice é Minas Gerais(MG) acompanhado de São Paulo(SP), e os estados com
menor media de indice são Distrito Federal (DF) seguido de Roraima (RR)

\includegraphics{Rinse_files/figure-latex/plotando grafico por estado echo:false-1.pdf}

\textbf{Total de alunos por estado}

Plotando nosso grafico de barras, identificamos que o estado com maior
número de alunos é o estado de Sào Paulo (SP) seguido de Minas Gerais
(MG).

Com base nos dados, conseguimos identificar que a maior parte das
familias com filhos em escolas estão em São Paulo, também identificamos
uma maior demanda de professores e necessidade de investimento em edução
nesse estado,a principio apelas pelo número de crianças

\includegraphics{Rinse_files/figure-latex/plotando grafico de número de alunos-1.pdf}

\textbf{Resumo estatistico do estado de SP e MG}

São Paulo

\begin{verbatim}
##    MEDIA_INSE   
##  Min.   :0.000  
##  1st Qu.:0.000  
##  Median :5.080  
##  Mean   :3.131  
##  3rd Qu.:5.310  
##  Max.   :6.040
\end{verbatim}

Minas Gerais

\begin{verbatim}
##    MEDIA_INSE   
##  Min.   :0.000  
##  1st Qu.:0.000  
##  Median :4.610  
##  Mean   :3.076  
##  3rd Qu.:5.030  
##  Max.   :6.340
\end{verbatim}

\end{document}
